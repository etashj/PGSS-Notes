\documentclass[11pt,letterpaper]{article}
\setlength{\marginparwidth}{2cm}
\usepackage[english]{babel}
\usepackage[utf8x]{inputenc}
\usepackage{amsmath}
\usepackage{amssymb}
\usepackage{graphicx}
\usepackage{tikz}
\usepackage{textcomp}
\usepackage{titlesec}
\usepackage[colorinlistoftodos]{todonotes}
\raggedbottom

\titlespacing*{\subsubsection}
{0pt}{1ex plus 1ex minus .2ex}{0.8ex plus .2ex}

\def\changemargin#1#2{\list{}{\rightmargin#2\leftmargin#1}\item}
\let\endchangemargin=\endlist

\author{Etash Jhanji\\\small Collaborators: Micheal Huang}
\title{PGSS:\ Math Finance HW 6}
\date{}

\begin{document}
\maketitle
\begin{enumerate}
    \item\begin{align*}
            P_0^A &= \frac{1000}{\left(1+\frac{0.035}{12}\right)^{12}}\\
            P_0^A &= \$ 965.65\\
        \end{align*}
        \begin{align*}
            P_0^{A\!+\!B} &= 1000\cdot\frac{\left(1+\frac{f(1,5)}{12}\right)^{48}}{\left(1+\frac{0.05}{12}\right)^{60}}\\
        \end{align*}
        \begin{align*}
            0 &= P_0^A - P_0^{A\!+\!B}
        \end{align*}
        \begin{align*}
            f(s,T) &= 12{\left( \left[ \frac{(1+ \frac{r(0,T)}{12})^T}{(1+ \frac{r(0,s)}{12})^s} \right]^{\frac{1}{T-s}}-1 \right)}\\
            f(1,5) &= 12{\left( \left[ \frac{(1+ \frac{r(0,5)}{12})^5}{(1+ \frac{r(0,1)}{12})^1} \right]^{\frac{1}{5-1}}-1 \right)}\\
            f(1,5) &= 0.05375 \text{ or } 5.375\%
        \end{align*}
    \item \begin{enumerate}
        \item At $t=0$ take a loan of $P_0^F$ where $F = \$10000$ and make a deposit of the same amount over the interval $\left[0,\frac{11}{12}\right]$. An interest rate $r(0,1)$ is applied to the lean and a rate $r(0,\frac{11}{12})$ is applied to the deposit. Finally, at $t=\frac{11}{12}$ deposit the the amount $P_0^F$ over $\left[\frac{11}{12}, 12\right]$ to grow a loan to \$10,000 over a year and a deposit to the same amount plus $V_1$. 
        \item \begin{align*}
            V_0 &= \frac{10000}{\left(1+\frac{0.04}{12}\right)^{12}} - \frac{10000}{\left(1+\frac{0.039}{12}\right)^{11}}\\
            V_0 &= - \$ 40.84\\
        \end{align*}
        Not sure why this value is negative, perhaps the security pays the investor so when represented in loans and deposits it comes out to be negative? 
    \end{enumerate}
\end{enumerate}
\end{document}