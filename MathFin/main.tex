\documentclass[12pt,letterpaper, twocolumn]{article}
\setlength{\marginparwidth}{2cm}
\usepackage[english]{babel}
\usepackage[utf8x]{inputenc}
\usepackage{amsmath}
\usepackage{amssymb}
\usepackage{graphicx}
\usepackage{tikz}
\usepackage{textcomp}
\usepackage{titlesec}
\usepackage[colorinlistoftodos]{todonotes}

\titlespacing*{\subsubsection}
{0pt}{1ex plus 1ex minus .2ex}{0.8ex plus .2ex}

\def\changemargin#1#2{\list{}{\rightmargin#2\leftmargin#1}\item}
\let\endchangemargin=\endlist

\author{Etash Jhanji}
\title{PGSS:\ Math Finance Notes}
\date{}

\begin{document}
\maketitle


\section{Asset Classes}
The class will focus on assets which are indistinguishable from each other in terms of uniqueness (i.e.\ one share of a stock is identical to another). \! indicates a topic we will focus on and? indicates one that may be applicable. 
    \begin{itemize}
        \setlength\itemsep{0.5em}
        \item Real Estate
        \item[!] Cash/Currencies
        \item[!] Stocks \textrightarrow{} Equities
        \item Artwork \textrightarrow{} Collectibles 
        \item Vehicles \textrightarrow{} Durable goods
        \item[!] Certificates of Deposit (CDs) \& Bonds \textrightarrow{} Fixed Income
        \item[?] Cryptocurrencies
        \item NFTs
        \item[?] Precious metals \textrightarrow{} Commodities
    \end{itemize}

\section{Trading}
\begin{itemize}
    \setlength\itemsep{0.25em}
    \item The lowest selling price on a share is the \textbf{Asking Price}. 
    \item The highest buying price on a share is the \textbf{Bid Price}. 
    \item The \textbf{Bid-Ask spread} is the difference between bid price and ask price. 
    \item The bid-ask spread tends to be smaller in actively traded, larger stocks. 
\end{itemize}
\subsubsection*{Derivative Securities}
Securities that derive value from the values of other assets. 
\subsubsection*{Forward Contract}
An agreement on the parameters of an exchange at a specific date before that date. 
\subsubsection*{Call/Stock Options}
Gives a person the right (not an obligation) to sell an assets for a particular price on a particular day. 
\subsubsection*{Put Options}
Gives the right (but not the obligation) to sell an asset. 

\section{Mathematical Models of Financial Markets}
\subsection*{Assumptions}
\begin{enumerate}
    \item Single price to buy or sell (no bid-ask spread)
    \item Can buy or sell any amount without moving the price
    \item No trading fees
    \item No taxes
\end{enumerate} 

\section{Interest}
\begin{itemize}
\item Borrow some amount (\textbf{Principal})
\item Pay back at some later datePay an extra amount (\textbf{Interest})
\item Borrow $P$, repay $P+I$ at time $T>0$ (\textbf{maturity}) at $t=0$
\end{itemize}
\subsection*{Why do banks charge interest on loans? }
\begin{enumerate}
    \item To compensate for default risk
    \item To compensate for inflation risk
    \item `Time value of money'
    \item Compensate for `opportunity cost'
\end{enumerate}
\subsection*{Why do banks pay interest on deposits? }
Investors demand it
\subsection*{Why do banks accept deposits? }
To borrow at a low interest rate and loan at a higher rate using your money
\subsection{Interest Payments}
\subsubsection{Simple Interest}
\begin{itemize}
    \item Proportional to the size of the principal
    \item Proportional to the loaning period length
\end{itemize}
Given a loan with principal $P$, maturity $T$, and interest rate $r$ where maturity is in years: principal is in dollars (or some other currency), and the interest rate is a rate in units $\frac{1}{\text{year}}$
The formula for the total interest paid would be as follows: 
\begin{align*}
    I &= r \text{[$\frac{1}{\text{year}}$]} \cdot P \text{[\$]} \cdot T \text{[years]}\\
    I &= rPT \text{[\$]}
\end{align*}
The total payment would be
\begin{align*}
    P + I &= P + rPT\\
    P + I &= P(1+rT)
\end{align*}
\textit{Note} These principles assume simple interest

Interest is earned uniformly throughout the loan. 
i.e. $A_0$ borrowed at $t=0$ grows to $A_t=(1+rT)A_0$

\subsubsection{Compound Interest}
Interest is \textit{compounded} to the interest during the loan. 
\\\\\\\textbf{Example}\\
\$100 borrowed for 1 year at interest rate $r$ compounded quarterly (3 months)
\begin{align*}
    A_0 &= \text{\$}100\\
    t &= \frac{1}{4}\\
    A_{\frac{1}{4}} &= 100 + (\frac{100}{4})r\\
    A_{\frac{1}{4}} &= 100(1 + \frac{r}{4})\\
    \text{The above}& \text{ becomes the new principal}\\
    t &= \frac{1}{2}\\
    A_{\frac{1}{2}} &= [100(1 + \frac{r}{4})] + \frac{r}{4}[100(1 + \frac{r}{4})]\\
    A_{\frac{1}{2}} &= (1 + \frac{r}{4})[100(1 + \frac{r}{4})]\\
    A_{\frac{1}{2}} &= 100(1 + \frac{r}{4})^2\\
    \text{The above}& \text{ becomes the new principal again}\\
    t &= \frac{3}{4}\\
    A_{\frac{3}{4}} &= (1 + \frac{r}{4})[100(1 + \frac{r}{4})^2]\\
    A_{\frac{3}{4}} &= 100(1 + \frac{r}{4})^3\\
    \text{The above}& \text{ becomes the new principal again}\\
    A_{1} &= 100(1 + \frac{r}{4})^4\\
    \text{More}&\text{ generally}\\
    A_T &= A_0(1+\frac{r}{m})^{mT}\\
    \text{Where $m$}& \text{ is the number of compounding} \\&\text{periods per year}
\end{align*}
Compound interest allows for exponential growth of interest. 
\\\\
The \textit{quoted} interest rate depends on the compounding convention even if payments are the same. 

\subsubsection{Negative Rates}
\begin{itemize}
    \item Interests rates can be negative
    \item But $(1+\frac{r}{12})^{12T}$ must be positive
    \item Thus mathematically $r>-12$
\end{itemize}

\subsection{The time value of money}
If you were to receive \$1000 one year from now, what is the value of that future payment today? 
\\\\
\textit{Note} A deposit of $\frac{1000}{(1+\frac{r}{12})^{12}}$ it will grow to $(\frac{1000}{(1+\frac{r}{12})^{12}})(1+\frac{r}{12})^{12}$ or \$1000 at $t=1$
\\\\
We say $\frac{1000}{(1+\frac{r}{12})^{12}}$ is the \textbf{present value} of 1000 to be paid one year from now. Thus you can take a loan for that amount and pay off the loan with interest using the credit from the \$1000 you are to receive in a year (it all cancels). 
\\\\
For the present value of several payments,  add up the values of each individual payment

\subsection{Fixed Income Securities}
\subsubsection{Zero Coupon Bonds}
\textbf{Zero Coupon Bonds} make a single payment at a single time. They pay a face value $f$ at the maturity $T>0$. 

If we have a bank with interest rate $r$ the discounted present value of the ZCB is $P_0 = \frac{F}{(1+\frac{r}{12})^12T}$. 

If we deposit $A_0 = \frac{F}{(1+\frac{r}{12})^12T}$ at $t=0$ until $T_1$ then $A_T = F$. 

\subsubsection{Annuities}
An \textbf{annuity} is a series of same-sized payments made at regular intervals. It will make payments of \$$A$, $m$ times per year, for $T$ years
\\\\
\textit{Example}
An annuity makes payments of \$200 for 2 years. To rewrite it, you could make it a sum of ZCBs, each being a month apart from each other in 24 fixed payments (emulating the annuity). 
\\\\
This makes the present value of an annuity be
 \[P_0^A = \sum_{i=1}^{24}\frac{200}{(1+\frac{r}{12})^i}\]


\textit{Example}
An annuity makes payments of \$500 quarterly for 1 year. The net present value is $P_0^{A_2}= \frac{500}{(1+\frac{r}{12})^3} + \frac{500}{(1+\frac{r}{12})^6} + \frac{500}{(1+\frac{r}{12})^9} + \frac{500}{(1+\frac{r}{12})^{12}}$ or $\sum_{i=1}^{4}\frac{500}{(1+\frac{r}{12})^{3i}}$

\subsection{Untitled}
\subsubsection{Coupon Bonds}
\end{document}