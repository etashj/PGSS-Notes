\documentclass[12pt,letterpaper, twocolumn]{article}
\setlength{\marginparwidth}{2cm}
\usepackage[english]{babel}
\usepackage[utf8x]{inputenc}
\usepackage{amsmath}
\usepackage{amssymb}
\usepackage{graphicx}
\usepackage{tikz}
\usepackage{textcomp}
\usepackage{titlesec}
\usepackage[colorinlistoftodos]{todonotes}

\titlespacing*{\subsubsection}
{0pt}{1ex plus 1ex minus .2ex}{0.8ex plus .2ex}

\def\changemargin#1#2{\list{}{\rightmargin#2\leftmargin#1}\item}
\let\endchangemargin=\endlist

\author{Etash Jhanji}
\title{PGSS:\ Math Finance Notes}
\date{}

\begin{document}
\maketitle


\section{Asset Classes}
The class will focus on assets which are indistinguishable from each other in terms of uniqueness (i.e.\ one share of a stock is identical to another). \! indicates a topic we will focus on and? indicates one that may be applicable. 
    \begin{itemize}
        \setlength\itemsep{0.5em}
        \item Real Estate
        \item[!] Cash/Currencies
        \item[!] Stocks \textrightarrow{} Equities
        \item Artwork \textrightarrow{} Collectibles 
        \item Vehicles \textrightarrow{} Durable goods
        \item[!] Certificates of Deposit (CDs) \& Bonds \textrightarrow{} Fixed Income
        \item[?] Cryptocurrencies
        \item NFTs
        \item[?] Precious metals \textrightarrow{} Commodities
    \end{itemize}

\section{Trading}
\begin{itemize}
    \setlength\itemsep{0.25em}
    \item The lowest selling price on a share is the \textbf{Asking Price}. 
    \item The highest buying price on a share is the \textbf{Bid Price}. 
    \item The \textbf{Bid-Ask spread} is the difference between bid price and ask price. 
    \item The bid-ask spread tends to be smaller in actively traded, larger stocks. 
\end{itemize}
\subsubsection*{Derivative Securities}
Securities that derive value from the values of other assets. 
\subsubsection*{Forward Contract}
An agreement on the parameters of an exchange at a specific date before that date. 
\subsubsection*{Call/Stock Options}
Gives a person the right (not an obligation) to sell an assets for a particular price on a particular day. 
\subsubsection*{Put Options}
Gives the right (but not the obligation) to sell an asset. 

\section{Mathematical Models of Financial Markets}
\subsection*{Assumptions}
\begin{enumerate}
    \item Single price to buy or sell (no bid-ask spread)
    \item Can buy or sell any amount without moving the price
    \item No trading fees
    \item No taxes
\end{enumerate} 

\section{Interest}
\begin{itemize}
\item Borrow some amount (\textbf{Principal})
\item Pay back at some later datePay an extra amount (\textbf{Interest})
\item Borrow $P$, repay $P+I$ at time $T>0$ (\textbf{maturity}) at $t=0$
\end{itemize}
\subsection*{Why do banks charge interest on loans? }
\begin{enumerate}
    \item To compensate for default risk
    \item To compensate for inflation risk
    \item `Time value of money'
    \item Compensate for `opportunity cost'
\end{enumerate}
\subsection*{Why do banks pay interest on deposits? }
Investors demand it
\subsection*{Why do banks accept deposits? }
To borrow at a low interest rate and loan at a higher rate using your money
\subsection{Interest Payments}
\subsubsection{Simple Interest}
\begin{itemize}
    \item Proportional to the size of the principal
    \item Proportional to the loaning period length
\end{itemize}
Given a loan with principal $P$, maturity $T$, and interest rate $r$ where maturity is in years: principal is in dollars (or some other currency), and the interest rate is a rate in units $\frac{1}{\text{year}}$
The formula for the total interest paid would be as follows: 
\begin{align*}
    I &= r \text{[$\frac{1}{\text{year}}$]} \cdot P \text{[\$]} \cdot T \text{[years]}\\
    I &= rPT \text{[\$]}
\end{align*}
The total payment would be
\begin{align*}
    P + I &= P + rPT\\
    P + I &= P(1+rT)
\end{align*}
\textit{Note} These principles assume simple interest

Interest is earned uniformly throughout the loan. 
i.e. $A_0$ borrowed at $t=0$ grows to $A_t=(1+rT)A_0$

\subsubsection{Compound Interest}
Interest is \textit{compounded} to the interest during the loan. 
\\\\\\\textbf{Example}\\
\$100 borrowed for 1 year at interest rate $r$ compounded quarterly (3 months)
\begin{align*}
    A_0 &= \text{\$}100\\
    t &= \frac{1}{4}\\
    A_{\frac{1}{4}} &= 100 + (\frac{100}{4})r\\
    A_{\frac{1}{4}} &= 100(1 + \frac{r}{4})\\
    \text{The above}& \text{ becomes the new principal}\\
    t &= \frac{1}{2}\\
    A_{\frac{1}{2}} &= [100(1 + \frac{r}{4})] + \frac{r}{4}[100(1 + \frac{r}{4})]\\
    A_{\frac{1}{2}} &= (1 + \frac{r}{4})[100(1 + \frac{r}{4})]\\
    A_{\frac{1}{2}} &= 100(1 + \frac{r}{4})^2\\
    \text{The above}& \text{ becomes the new principal again}\\
    t &= \frac{3}{4}\\
    A_{\frac{3}{4}} &= (1 + \frac{r}{4})[100(1 + \frac{r}{4})^2]\\
    A_{\frac{3}{4}} &= 100(1 + \frac{r}{4})^3\\
    \text{The above}& \text{ becomes the new principal again}\\
    A_{1} &= 100(1 + \frac{r}{4})^4\\
    \text{More}&\text{ generally}\\
    A_T &= A_0(1+\frac{r}{m})^{mT}\\
    \text{Where $m$}& \text{ is the number of compounding} \\&\text{periods per year}
\end{align*}
Compound interest allows for exponential growth of interest. 
\\\\
The \textit{quoted} interest rate depends on the compounding convention even if payments are the same. 

\subsubsection{Negative Rates}
\begin{itemize}
    \item Interests rates can be negative
    \item But $(1+\frac{r}{12})^{12T}$ must be positive
    \item Thus mathematically $r>-12$
\end{itemize}

\subsection{The time value of money}
If you were to receive \$1000 one year from now, what is the value of that future payment today? 
\\\\
\textit{Note} A deposit of $\frac{1000}{(1+\frac{r}{12})^{12}}$ it will grow to $(\frac{1000}{(1+\frac{r}{12})^{12}})(1+\frac{r}{12})^{12}$ or \$1000 at $t=1$
\\\\
We say $\frac{1000}{(1+\frac{r}{12})^{12}}$ is the \textbf{present value} of 1000 to be paid one year from now. Thus you can take a loan for that amount and pay off the loan with interest using the credit from the \$1000 you are to receive in a year (it all cancels). 
\\\\
For the present value of several payments,  add up the values of each individual payment

\subsection{as it Relates to Maturities}
Interest rates depend on maturity. 
\begin{itemize}
    \item if we make a deposit from $t=0$ to $t=1$ or from $t=0$ to $t=5$
    \item Interest rates will be different
    \item The notation is for maturity $T$ is $r=r(T)$
    \item e.g. Two year ZCB with a face of \$1000
    \begin{align*}
        P^{Z} = \frac{F}{(1+\frac{r(2)(12)}{12})^{12\cdot 2}}\\
        P^{Z} = \frac{1000}{(1+\frac{r(2)(12)}{12})^{24}}\\
    \end{align*}
\end{itemize}
An annuity makes monthly payments of \$100 for 5 years. 
\begin{align*}
    P^A&=\frac{100}{(1+\frac{r{1/12}}{12})^1} + \dots + \frac{100}{(1+\frac{r{5}}{12})^12\cdot5}
\end{align*}
Typically $r(t)$ follows a logarithmic function where longer periods require higher interest. However, some \textit{inverted yield curves} decrease as time increases. However, interest rates can change, thus we use $r(s,t)$ to denote the interest rate agreed to at time $s$ for deposits over the interval $[s, t]$

So if we have at $t=0$, the price of a ZCB with a face value of \$1000 and maturity of 1 year will be
\begin{align*}
    P_0 &= \frac{1000}{(1+\frac{r(0,1)}{12})^{12}}
\end{align*}
The price of the same bond at time $s=\frac{1}{12}$ will be based on a deposit over the interval $[\frac{1}{12},1]$ or $[s,1]$ with interest $r(\frac{1}{12},1)$. The deposit of $A_{\frac{1}{12}}$ at time $\frac{1}{12}$ will grow to 
\begin{align*}
    A_1 &= A_{\frac{1}{12}}\left(1+\frac{r(\frac{1}{12},1)}{12}\right)^{11} = 1000
\end{align*}
\begin{align*}
    P_{\frac{1}{12}} &= A_{\frac{1}{12}}
\end{align*}

The price of a bond at time $s$ with maturity $T$ is 
\[
    \frac{F}{\left( 1+ \frac{r(s,t)}{12} \right)^{12(T-s)}}
\]
Typically interest rates are positive. Bond prices and interest rates are inversely related (i.e. move in opposite directions).

\subsection{Forward Interest Rates}
A interest rate agreed to at a time 0 for a deposit or a loan at time $s>0$ with maturity $T>s>0$. This rate is denoted by $f(s, T)$. Although $r(s, T)$ is not known at time 0, $f(s, T)$ is, although they may be the same at time $s$ by chance.  


\section{Fixed Income Securities}
\subsection{Zero Coupon Bonds}
\textbf{Zero Coupon Bonds} make a single payment at a single time. They pay a face value $f$ at the maturity $T>0$. 

If we have a bank with interest rate $r$ the discounted present value of the ZCB is $P_0 = \frac{F}{(1+\frac{r}{12})^12T}$. 

If we deposit $A_0 = \frac{F}{(1+\frac{r}{12})^12T}$ at $t=0$ until $T_1$ then $A_T = F$. 

\subsection{Annuities}
An \textbf{annuity} is a series of same-sized payments made at regular intervals. It will make payments of \$$A$, $m$ times per year, for $T$ years
\\\\
\textit{Example}
An annuity makes payments of \$200 for 2 years. To rewrite it, you could make it a sum of ZCBs, each being a month apart from each other in 24 fixed payments (emulating the annuity). 
\\\\
This makes the present value of an annuity be
 \[P_0^A = \sum_{i=1}^{24}\frac{F}{(1+\frac{r}{12})^i}\]


\textit{Example}
An annuity makes payments of \$500 quarterly for 1 year. The net present value is $P_0^{A_2}= \frac{500}{(1+\frac{r}{12})^3} + \frac{500}{(1+\frac{r}{12})^6} + \frac{500}{(1+\frac{r}{12})^9} + \frac{500}{(1+\frac{r}{12})^{12}}$ or $\sum_{i=1}^{4}\frac{500}{(1+\frac{r}{12})^{3i}}$

\section{Portfolios}
A \textbf{portfolio} is a collection of assets and and rules for trading among them. In mathematical finance it is generally thought of as a situation fo assets bought and sold at specific times. 

If we were to use $X$ as a label for a portfolio, then $X_t$ is the value of the portfolio at time $t$. The value $x_0$ is called the \textbf{initial capital} of the portfolio. 

\subsection{Arbitrage}
A potential for risk free profit. 
An \textbf{arbitrage portfolio} is defines as a portfolio $X$ such that $X_0 = 0$ and $X_T \geq 0$ with $X_T > 0$ is possible. It is, in essence, free lunch and as we know there are no free lunches (THANSTAAFL). 

Although arbitrage is possible in small and difficult to obtain amounts in real markets, we will assume they don't exist in our market. 

\subsubsection{Arbitrage of ZCBs}
What is the arbitrage price of a zero coupon bond? Given face value $F$, maturity $T$, interest rate $r$
\begin{align*}P&=\frac{F}{(1+\frac{r}{12})^{12T}}\end{align*}

\subsubsection*{Example}
\textbf{Buying.} 
\begin{align*}
    F &= \$1000\\
    T&= 5\text{ years}\\
    r&= 0.04\\
    P&= \frac{1000}{(1+\frac{0.04}{12})^{12\cdot 5}}\\
    P&= \$819.0031
\end{align*}
\$819 is the arbitrage-free price of the ZCB.
\\\\
Suppose we can buy or sell the bond for $\hat{P}=815<P$. In Portfolio $X$ we can borrow 815 from the bank and buy the bond. 
\begin{align*}
    X_0 &= [\text{bond}] - [\text{loan amount}]\\
    X_0 &= 815-815\\
    X_0 &= 0\\
\end{align*}
\begin{align*}
    X_5 &= [\text{bond after maturity}] - [\text{loan amount}]\\
    X_5 &= 1000-815(1+\frac{.04}{12})^{12\cdot5}\\
    X_5 &= \$4.89\\
    X_5 &> 0
\end{align*}
After 5 years you have made a profit of \$4.89, creating an arbitrage since the bond's purchase price was lower than its present value. 

\subsubsection*{Example}
\textbf{Selling.} 
Suppose we can buy or sell the bond for $P=\$825$. In portfolio $X$ we can sell the bond and deposit the proceeds in the bank. 
\begin{align*}
    X_0 &= -[\text{bond}] + [\text{deposit amount}]\\
    X_0 &= -825-825\\
    X_0 &= 0\\
\end{align*}
\begin{align*}
    X_5 &= -[\text{mature bond}] + [\text{deposit and interest}]\\
    X_5 &= -1000+825(1+\frac{.04}{12})^{12\cdot5}\\
    X_5 &= \$7.32\\
    X_5 &> 0
\end{align*}

If $\hat{P} \neq P$ then there is an Arbitrage opportunity. THe arbitrage free price is $\hat{P} = P$ where $P$ is the discounted present value. 

\subsection{The Law of One Price}
Suppose there is no arbitrage and you have two portfolios, $X$ and $Y$, of fixed income securities that make all the same payments at all the same times. Then $X_0 = Y_0$.
\\\\\textbf{Proof}\\
Suppose $X_0 \neq Y_0$ ($X_0 < Y_0$)\\
Then you should buy $X$ and sell $Y$\\
Deposit $Y_0 - X_0 <0$ in the bank until $T>0$\\
Let this portfolio be called $Z$. \\
\begin{align*}
    Z_0 &= X_0 - Y_0 + (Y_0 - X_0)\\
    Z_0 &= 0
\end{align*}
There is no net cash flow as they all cancel except of the deposity is $>0$. 

\subsection{Coupon Bonds}
Has a face value $F$, a maturity $T$, maturity $m$, and a coupon rate $q$. The coupon rate is the rate of interest paid on the face value.
\\It makes payments as follows:
\begin{align*}
    C&= F \cdot \frac{q}{m} \text{ at } \frac{1}{m}, \frac{2}{m}, \dots\\
\end{align*}
At $t=T$ it pays $C+F = F(1+\frac{q}{m})$.
\\\\
This can be replicated with a portfolio of ZCBs with face values $C$ and maturities $\frac{1}{m}, \frac{2}{m}, \dots$ and a ZCB with face value $F+C$ maturing at $T$. \\

The arbitrage free price of a coupon bnd must be the sum of the arbitrage free prices of the ZCBs. 

\begin{align*}
    P^{\text{CB}} &= \frac{C}{(1+\frac{r}{m})^1} + \dots + \frac{C}{(1+\frac{r}{m})^{mT-1}} + \frac{F+C}{(1+\frac{r}{m})^{mT}}\\
    P^{\text{CB}} &= \sum_{i=1}^{mT}\frac{C}{(1+\frac{r}{m})^i} + \frac{F}{(1+\frac{r}{m})^{mT}}
\end{align*}

To find an arbitrage free price find a portfolio $X$ of securities who's prices are known that replicates the payments of the securities. The arbitrage free price is the initial capital of the portfolio. We say $X$ is a \textbf{replicating portfolio} of the security. Because of the \textit{Law of One Price} the replicating portfolios must have the same initial capital. 

\subsection*{Example of Forward Interest Portfolios}
Take portfolio $X$ which is composed of a loan (borrowing) 
\[ \frac{1000}{\left( 1+ \frac{r(0,T)}{12} \right)^{12T}} \text{ over } [0,T] \]
such that $X_T = 1000$. 

Now take portfolio $Y$ which is a agreement to forward deposit 
\[ \frac{1000}{\left( 1+ \frac{f(s,T)}{12} \right)^{12(T-s)}} \text{ over } [s,T] \]
Thus depositing 
\[ \frac{1000}{\left( 1+ \frac{f(s,T)}{12} \right)^{12(T-s)}\left( 1+ \frac{r(0,s)}{12} \right)^{12s}} \text{ over } [0,T] \]
At time $s$ the deposit will grow to
\[ \frac{1000}{\left( 1+ \frac{f(s,T)}{12} \right)^{12(T-s)}}\]
and at time $T$ it will grow to exactly \$1000. 
So $Y_T = 1000$, meaning by the Law of One Price since $X_T = Y_T=1000$ then $X_0 = Y_0$, and using some algebra we can derive
\[
    f(s,T) = 12{\left( \left[ \frac{(1+ \frac{f(0,T)}{12})^T}{(1+ \frac{f(0,s)}{12})^s} \right]^{\frac{1}{T-s}}-1 \right)}
\]
\end{document}