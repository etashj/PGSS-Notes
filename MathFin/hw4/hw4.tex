\documentclass[11pt,letterpaper]{article}
\setlength{\marginparwidth}{2cm}
\usepackage[english]{babel}
\usepackage[utf8x]{inputenc}
\usepackage{amsmath}
\usepackage{amssymb}
\usepackage{graphicx}
\usepackage{tikz}
\usepackage{textcomp}
\usepackage{titlesec}
\usepackage[colorinlistoftodos]{todonotes}
\raggedbottom

\titlespacing*{\subsubsection}
{0pt}{1ex plus 1ex minus .2ex}{0.8ex plus .2ex}

\def\changemargin#1#2{\list{}{\rightmargin#2\leftmargin#1}\item}
\let\endchangemargin=\endlist

\author{Etash Jhanji\\\small Collaborators: Rohan Dalal}
\title{PGSS:\ Math Finance HW 4}
\date{}

\begin{document}
\maketitle
\begin{enumerate}
    \item \begin{enumerate}
        \item Because the bond makes payments of \$125 it can be represented as 5 payments of \$25
        \begin{align*}
            P&=5(P_0^A) = 5(97.67)\\
            P&=\$448.35
        \end{align*}
        \item\begin{align*}
            P&=10(P_0^Z) = 10(957)\\
            P&=\$9570
        \end{align*}
        \item Represented as ZCBs, payments should be at $\frac{1}{4}, \frac{1}{2}, \frac{3}{4}, 1$. 
        \begin{align*}
            c&=F\cdot\frac{q}{m}\\
            c&=10000\cdot\frac{0.05}{4}\\
            c&=125
        \end{align*}
        The ZCB payments should be \$125 plus the face value at the end. 
        \begin{align*}
            F&=\sum_{i=1}^{4} \frac{125}{(1+\frac{0.05}{4})^{12/i}} + \frac{10000}{(1+\frac{0.05}{4})^{12}}\\
            F&=463.09+8615.09\\
            F&=9078.18\\
        \end{align*}
    \end{enumerate}
    \item \begin{align*}
        P&= \frac{500}{(1+\frac{.03}{12})^3} + \frac{500}{(1+\frac{.04}{12})^6} + \frac{500}{(1+\frac{.045}{12})^9} + \frac{500}{(1+\frac{.0475}{12})^{12}}\\
        P&= \(\)$1946.67
    \end{align*}
\end{enumerate}
\end{document}