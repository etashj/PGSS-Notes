\documentclass[12pt,letterpaper]{article}
\setlength{\marginparwidth}{2cm}
\usepackage[english]{babel}
\usepackage[utf8x]{inputenc}
\usepackage{amsmath}
\usepackage{amssymb}
\usepackage{graphicx}
\usepackage{tikz}
\usepackage{textcomp}
\usepackage{titlesec}
\usepackage[colorinlistoftodos]{todonotes}
\raggedbottom

\titlespacing*{\subsubsection}
{0pt}{1ex plus 1ex minus .2ex}{0.8ex plus .2ex}

\def\changemargin#1#2{\list{}{\rightmargin#2\leftmargin#1}\item}
\let\endchangemargin=\endlist

\author{Etash Jhanji}
\title{PGSS:\ Math Finance HW 3}
\date{}

\begin{document}
\maketitle
\begin{enumerate}
    \item \begin{enumerate}
        \item It makes payments of \$25 at time $\frac{1}{2}$, \$50 at time $1$, \$75 at time $\frac{3}{2}$, and \$100 at time $2$. 
        \item \begin{align*}
            P_1 = \frac{25}{(1+\frac{0.06}{12})^{24}} &= \$22.18\\
            P_2 = \frac{25}{(1+\frac{0.06}{12})^{18}} &= \$22.85\\
            P_3 = \frac{25}{(1+\frac{0.06}{12})^{12}} &= \$23.55\\
            P_4 = \frac{25}{(1+\frac{0.06}{12})^{6}} &= \$24.26\\
            F_P = \frac{1000}{(1+\frac{0.06}{12})^{24}} &= \$887.19\\
            \sum{P} + F_P &= \$980.03
        \end{align*}
        \item To reconstruct this portfolio using zero coupon bonds, you need one bond of \$25 maturing each quarter with the last one (after 2 years) being of value \$1025. 
    \end{enumerate}
\end{enumerate}
\end{document}