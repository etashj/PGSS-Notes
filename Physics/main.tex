\documentclass[12pt,letterpaper, twocolumn]{article}
\setlength{\marginparwidth}{2cm}
\usepackage[english]{babel}
\usepackage[utf8x]{inputenc}
\usepackage{amsmath}
\usepackage{amssymb}
\usepackage{graphicx}
\usepackage{tikz}
\usepackage{mathtools}
\usepackage[colorinlistoftodos]{todonotes}

\author{Etash Jhanji}
\title{PGSS:\@ Physics Notes}
\date{}

\begin{document}
\maketitle


\section{Newtonian Physics}
    \subsection{Position, Velocity, Acceleration, and Forces}
    Given that $x(t)$ is position as a function of time, $v(t)$ is velocity against time, and $a(t)$ is acceleration as a function of time. 
    \begin{align*}
        v(t) &= \frac{\Delta x(t)}{\Delta t} = \frac{dx}{dt}\\
        a(t) &= \frac{\Delta v(t)}{\Delta t} = \frac{dv}{dt}\\
        a(t) &= \frac{d^2x}{dt^2}
    \end{align*}
    \begin{itemize}
        \item[] Any change in velocity, or the rate of change of position, is acceleration and must be due to a force (Newton's 1st)
    \end{itemize}
    \begin{align*}
        F&=ma\\
        a &= \frac{F}{m}
    \end{align*}
    \begin{itemize}
        \item[] Mass ($m$) is inversely proportional to acceleration and force ($F$) is directly proportional
        \item[] Thus more force means greater acceleration and more mass means less acceleration
    \end{itemize}
    \begin{align*}
        F&=G_n\frac{m_1m_2}{(x_1-x_2)^2}
    \end{align*}
    Where $G_n$ is the gravitation constant, $m$ is the mass of each object, and $x$ is the position of each object on a single axis, the above equation determines the force of attraction between two objects.
    
    \subsection{Momentum and Energy}
    Conserved quantities which make problems simpler to solve.
    \begin{align*}
        p &= mv\\
        E &= \frac{1}{2}mv^2\\
        \Delta E &= 0 \\
        \sum_{P} E_i^P &= \sum_{P} E_f^P 
    \end{align*}

    \subsection{Example}
    \begin{itemize}
        \item[Q] Two particles of equal mass oppose each other exactly and travel at the same velocity that collide, producing two particles of equal masses that also oppose each other. One of the produced particles head $\frac{\pi}{4}$ radians elevated from the x-axis. What is the speed of the two produced particles?
        \item[A] The original particles have opposite momenta which cancel each other out giving the system a momentum of 0. Thus the produced particles will also have to have the same component momenta to cancel each other out, defined as $p=mv_f$ where $v_f$ is the final velocity of the produced particles. Using energy conservation we can also say energy is conserved from each original particle and that $v_f = v_i$. 
    \end{itemize}

\section{Coordinates and Coordinate Changes}
\begin{itemize}
    \item[] Frame of reference can be changed between particles to make algebraic calculations easier. 
    \item[] Each frame of reference has different numbers for calculations
    \item[] Well-defined relations between frames allow conversion
    \item[] Interrelated frames create an \textit{equivalence class}. 
\end{itemize}
\subsection{Moving and and relating frames}
\begin{itemize}
    \item Given two frames $s$ and $s'$ where over time the frame $s'$ moves in the positive x direction over time at a constant velocity 
    \item[Note] Visualize these two frames as a 2D set of axes with position ($x$) on the x-axis and time ($t$) on the y-axis. THey start off on top of each other at $t=0$ before diverging. 
    \item We know that $t=t'$ because time is a constant
    \item However we do not know the relation between $x$ and $x'$
    \item We can determine $x \rightarrow x'(t') = x'(t) = x(t) -v_s t$
    \item Considering $x$ and $t$ as vectors. 
    \begin{align*}
        \begin{pmatrix}
               x' \\
               t' 
             \end{pmatrix}
             &=  \begin{pmatrix}
                x \\
                t 
              \end{pmatrix}
              \begin{pmatrix}
                1 & -v_s \\
                0 & 1 
              \end{pmatrix}
      \end{align*}
      \item Now given frames $x$, $x'$, and $x''$ where $x'$ moves relative to $x$ with a velocity of $v_1$ and $x''$ moves relative to $x'$ with a velocity of $v_2$
      \item To determine the position of $x''$ we can use the following
      \begin{align*}
        \begin{pmatrix}
            x''\\t''
        \end{pmatrix} &=
        \begin{pmatrix}
            1&-v_2\\0&1
        \end{pmatrix}
        \begin{pmatrix}
            1&-v_1\\0&1
        \end{pmatrix}
        \begin{pmatrix}
            v\\t
        \end{pmatrix}
        \\
        &=\begin{pmatrix}
            1&-v_1-v_2\\0&1
        \end{pmatrix}\begin{pmatrix}
            v\\t
        \end{pmatrix}\\
        &= \begin{pmatrix}
            1&-v_c\\0&1
        \end{pmatrix}\begin{pmatrix}
            v\\t
        \end{pmatrix}\\
        &= \begin{pmatrix}
            x-v_c t\\t
        \end{pmatrix}
    \end {align*}
      \item You can also write the matrix as a \textit{Galilean Transformation}
      \begin{align*}
        G(v_s) &= \begin{pmatrix}
            1&v_s\\0&1
        \end{pmatrix}\\
        G(v_2) \cdot G(v_1) &= G(v_1+v_2)
      \end{align*}
      \item In rotates frames where one frame is turned $\theta$ you can use the following equations
      \begin{align*}
        x' &=\cos\theta x + \sin\theta y\\
        y' &=\sin\theta x + \cos\theta y\\
        \begin{pmatrix}x'\\y'\end{pmatrix} &= \begin{pmatrix}\cos\theta&\sin\theta\\-\sin\theta&\cos\theta\end{pmatrix}\begin{pmatrix}x\\y\end{pmatrix}\\
        R(\theta) &=\begin{pmatrix}\cos\theta&\sin\theta\\-\sin\theta&\cos\theta\end{pmatrix}\\
        R(\theta_1) + R(\theta_2) &= R(\theta_1+\theta_2)
      \end{align*}
\end{itemize}

\subsection{How the physics changes in inertial frames}
\begin{align*}
    x'&={v_s}t\\
    F(|x_1-x_2|) &= F'(|x_1'-x_2'|)\\
    x_1' &= x_1-{v_s}t\\
    x_2' &= x_2-{v_s}t\\
    x_1' - x_2' &= x_1 - x_2\\
    m &= m'\\
    v_1' &= \frac{dx}{dt} = \frac{\Delta(x_1-{v_s}t)}{\Delta t}\\
    &= v_1-v_s\\
    a_1' &= \frac{\Delta v_1'}{\Delta t} = \frac{\Delta (v_1-v_s)}{\Delta t}\\
    &=\frac{\Delta v_1}{\Delta t} - \frac{\Delta v_s}{\Delta t} = a\\
    a_1' &= a\\
\end{align*}
\textit{Note} If $F=ma$ (and momentum and energy) applied in one frame it applies in all others. 

\subsection{Why it doesn't work}
\begin{itemize}
    \item Electromagnetism
        \item changing electromagnetic fields can alter forces
        \item Speed of light: $c' = c-v_s$ breaking the constant velocity of light
    \item Determinism
    \item General coordinate transformations
    \item Experimentally untrue (Michaelson and Morely)
\end{itemize}

\subsection{Linear Experiments}
\begin{itemize}
    \item \begin{align*}
        x&= ax' + bt'\\
        t&= ex'+ft'
    \end{align*}
    \item When $x'=0$, $x={v_s}t$
    \item When $x=0$, $x=-{v_s}t$
    \item OR \begin{align*}
        \begin{pmatrix}x\\t\end{pmatrix} &= \begin{pmatrix}x&b\\e&f\end{pmatrix}\begin{pmatrix}x'\\t'\end{pmatrix}
    \end{align*}
    \item \begin{align*}
        T(v_2)T(v_1) &= T(v_c)\\
        T(-v)T(v) &= T(0)
    \end{align*}
    \item The entire scenario is parametrized by $v_*$ \begin{align*}
        \begin{pmatrix}x\\t\end{pmatrix} &= \frac{1}{\sqrt{1-(\frac{v_1}{v_*})^2}}\begin{pmatrix}1&v_s\\\frac{v_s}{v_*^2} & 1\end{pmatrix}\begin{pmatrix}x'\\t'\end{pmatrix}\\
        \text{In Galilean: }v_* &= \infty\\
        x' &= v_{*}t' \rightarrow x = v_{*}t
    \end{align*}
\end{itemize}

\section{Matrices Crash Course}
\begin{align*}
    \begin{pmatrix}
        a_{11} & a_{12} \\
        a_{21} & a_{22} 
      \end{pmatrix}
      \begin{pmatrix}
        c_1\\c_2
      \end{pmatrix} &=
      \begin{pmatrix}
        a_{11}c_1+a_{12}c_2\\
        a_{21}c_1+a_{22}c_2
      \end{pmatrix}
      \\
      \begin{pmatrix}
        a_{11} & a_{12} \\
        a_{21} & a_{22} 
      \end{pmatrix}
      \begin{pmatrix}
        b_{11} & b_{12} \\
        b_{21} & b_{22} 
      \end{pmatrix} &=
      \begin{pmatrix}
        a_{11}b_{11} + a_{12}b_{21} & a_{11}b_{12} + a_{12}b_{22} \\
        a_{21}b_{11} + a_{22}b_{21} & a_{21}b_{12} + a_{22}b_{22} \\
      \end{pmatrix}
\end{align*}

\section{Principles of Relativity}
\begin{itemize}
    \item Given a flash of light at a point, the photons leave the point at a constant speed $c$
    \item If an observer is moving towards the light, it should appear that it is going faster than $c$ whereas moving away should indicate movement slower than $c$, however this is not the case 
\end{itemize}
\subsection{Units Analogy}
\begin{itemize}
    \item Two surveying contractors are surveying CMU
    \item One set of surveyors views the campus from a frame on the $x$ axis and the other from a rotated frame on the $x'$ axis
    \item The differing orientations of the frames cause the surveyors to measure different distances and angles with different coordinates
    \item Hypothetically company 1 is composed on engineers and 2 is chemists
    \item Lets say the surveyors use meters for x distance and miles for the y distance. 
    \item The physics student (our hero) determines a conversion factor $k$ to convert between the two frames and measurements. 
    \item $k=\frac{\text{m}}{\text{miles}}$
    \item Dist: $\sqrt{x^2+(ky)^2}=\sqrt{x'^2+(ky')^2}$
\end{itemize}
\subsection{For Relativity}
\begin{itemize}
    \item Instead of measuring t in seconds, we will make $[t]=[ct]$ so that we have units of meters and all coordinates are in meters. 
\end{itemize}
\subsection{Example}
\begin{align*}
    (ct, x, y, x) &\rightarrow (ct', x', y', z')\\
    y&=y'\\
    z&=z'
\end{align*}
Only x movement is the simplest case. 
\subsection{Light clock}
\begin{itemize}
    \item Given a clock which uses the speed of light against a mirror to measure a tick from light omission at time $A$ and reception at time $B$. 
    \item Put this on a rocket. The entire frame is now moving and the light instead of being up and down now moves to the right simultaneously in an externally observational frame
    \item In the rocket frame however, the clock is the same and $\Delta x=0$
    \item \begin{align*}
        t&=t'=0\\
        x_A&=0\\
        t_A&=0\\
        \text{In rocket}\\
        x'_A&=0\\
        t'_B&=2m\\
        \Delta x'&=0\\
        \Delta t'&=2m\\
        \text{Lab frame}\\
        \Delta x &> 0\\
        \Delta t &= 2\sqrt{1+\frac{\Delta x}{x}^2}\\
        \Delta t &> \Delta t'\\ 
    \end{align*}
    Moving clocks run slow. 

    \textit{Note} \begin{align*}
        \Delta t^2-\Delta x^2 &= 4(1+(\frac{\Delta x}{2})^2)-\Delta x^2\\
        &= 4
    \end{align*}
    \begin{align*}
        \Delta t'^2-\Delta x'^2&\equiv (\text{Interval})^2\\
        \Delta x^2-\Delta y^2&\equiv (\text{length})^2
    \end{align*}
    i.e. The sums form a hyperbola and  circle with a radius of length, respectively. 
\end{itemize}

\vfill\null
\subsection*{Lorenz Geometry}
When the interval ($\Delta t^2 - \Delta x^2$) is 
\begin{itemize}
    \item[$< 0$] it is `time-like'
    \item[$= 0$] it is `light-like'
    \item[$> 0$] it is `space-like'
\end{itemize}
This is a physical property of two events.

\textit{Note} in Euclidean Geometry a straight path is shorter than a curved path whereas in Lorenz Geometry a curved path is shorter than a straight path.

\subsection{Lorenz Transformations}
\begin{align*}
    \beta &= \frac{v}{c}\\
    \beta &= [0,1]\\
    \gamma &= \frac{1}{\sqrt{1-\beta^2}}\\
    \gamma &= [1,\infty)\\
    \begin{pmatrix}
    x\\t
    \end{pmatrix} &= \begin{pmatrix}\gamma & \beta \gamma \\ \beta \gamma & \gamma\end{pmatrix}\begin{pmatrix}x'\\t'\end{pmatrix}
\end{align*}

\subsection{Velocity Transformations}
In a Galilean transformation $v_x=v_x'+v_s$ and $v_y=v_y'$. 

In a Lorenz transformation, y velocity takes the form $\beta_y=\frac{\Delta y}{\Delta t} = \frac{\Delta y'}{\gamma \Delta t'} = \frac{1}{\gamma}\beta_y'$

\subsubsection*{The X-component}
\begin{align*}
    \beta_x &= \frac{\Delta x}{\Delta t}\\
    \beta_x &= \frac{\gamma \Delta x' + \beta \gamma \Delta t'}{ \beta \gamma \Delta x' + \gamma \Delta t'}\\
    \beta_x &= \frac{ \Delta x' + \beta  \Delta t'}{ \beta  \Delta x' +  \Delta t'}\\
    \beta_x &= \frac{\frac{\Delta x'}{\Delta t'} + \beta}{ \beta  \frac{\Delta x'}{\Delta t'} +  1}\\
    \beta_x &= \frac{\beta_x' + \beta}{1+\beta\beta_x'}
\end{align*}

\subsection{Applications of Lorenz Transformations}
\subsubsection*{The Train Paradox}
\begin{itemize}
    \item A train is moving at a relativistic speed
    \item There is a person on the train $o'$ who is moving at the same speed as the train and another observer $o$ who is stationary and the two line up
    \item Each one lights a bulb at the same time, and the lights cross at the same spot (where the observers are) in the train and stationary frames
    \item So which light bulb went off first? 
    \item First we consider the prime frame ($s'$)
    \begin{itemize}
        \item The light collides at $x'=0$ and $t'=0$
        \item Relative to the observer in the train, the bulbs are at rest, and the light moves at a constant speed, appearing as opposing 45 degree lines on a graph
    \end{itemize}
    \item And now the stationary, $s$, frame
    \begin{itemize}
        \item The bulbs are moving to this observer moving in parallel, yet he sees that they collide at the same time. 
        \item Observer $o$ sees the light from one light bulb go farther, thus it must have gone off first  
    \end{itemize}
\end{itemize}

\subsubsection*{The Twin Paradox}
\begin{itemize}
    \item Given two identical twins, one stays on Earth and the other goes on a relativistic journey
    \item Hypothetically the space twin goes out for $\tau$ years and spends the same time coming back.
    \item By the time the space twin returns he has aged $2\tau$ years
    \item How is the Earth twin aging? 
    \item \begin{align*}
        \begin{pmatrix}\Delta x\\ \Delta t\end{pmatrix} &= \begin{pmatrix}\gamma & \beta \gamma\\ \beta \gamma & \gamma\end{pmatrix}\begin{pmatrix}0\\2\tau\end{pmatrix}\\
        \Delta t_{t_2} &= 2\tau \cdot \gamma\\
        \tau &= 5\text{y}\\
        \gamma &= 10\\
        w_{v1} &= 10\text{y}\\
        w_{v2} &= 100\text{y}\\
    \end{align*}
\end{itemize}

\subsubsection*{The Barn \& Pole Paradox}
\begin{itemize}
    \item You have a pole of length 20 meters, and it is being moved at a relativistic speed ($\beta$ such that the $\gamma = 2$) towards a Barn with size 10 meters
    \item Can the pole fit into the barn?
    \item Observer $o$ sees the length contraction on the pole
    \item Observer $o'$ sees the length contraction on the barn as it moves towards them at $-\beta$
    \item Which observer is right? 
\end{itemize}

\section{Dynamics}
The why of motion. 

\begin{itemize}
    \item Classical Newtonian physics says when an object changes motion it is due to an external force on the particle acted by another object
    \item In other words, newtonian physics is a series of interactions in the form of forces
\end{itemize}

\subsection*{Classical Momentum}
$P=mv=m\beta$ as long as $\beta \ll 1$
\subsection*{New Momentum}
The goal is to define a momentum that is conserved at relativistic speeds. 
\subsubsection*{Units}
\begin{align*}
    [m]&=\text{mass}\\
    [P]=[m\beta]=[m]&=\text{mass}\\
    [E]=[\frac{1}{2}m\beta^2]=[m]&=\text{mass}\\
\end{align*}
\subsubsection*{For relativity}
\begin{align*}
    \vec{P} &= m\gamma\vec{\beta}\\
\end{align*}
Note the mass is constant and the Lorentz transformation is applied to the velocity ($\beta$). Relativistic momentum can also be represnted as a 4 vector
\[P=m\frac{1}{\Delta T}\begin{pmatrix}\Delta t\\\Delta x\\\Delta t\\\Delta z\end{pmatrix} = \begin{pmatrix}P_t\\\vec{P}\end{pmatrix}\]

Now we can define $P_t$ as a `Relativistic energy' which can be defined 

\[P_t \sim m+\frac{1}{2}m\beta^2\]

Which is mass plus classical kinetic energy(!). 
\begin{align*}
    E&=m+\frac{1}{2}m\beta^2\\
    E_c&=mc^2+\frac{1}{2}m\beta^2c^2\\
\end{align*}
\begin{align*}
    E&=\gamma m\\
    m^2 &= E^2 - \vec{P}^2\\
    E^2 &= m^2 + \vec{P}^2\\
    P&= \begin{pmatrix}E\\\vec{P}\end{pmatrix}
\end{align*}



\end{document}