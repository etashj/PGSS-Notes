\documentclass[12pt,letterpaper, twocolumn]{article}

\usepackage[english]{babel}
\usepackage[utf8x]{inputenc}
\usepackage{amsmath}
\usepackage{amssymb}
\usepackage{graphicx}
\usepackage{tikz}
\usepackage[colorinlistoftodos]{todonotes}
\usepackage{indentfirst}

\author{Etash Jhanji}
\title{PGSS: Computer Science Notes}
\date{}

\begin{document}
\maketitle


\section{Deterministic Finite Automata (DFAs)}

% \subsection{Simple Random Walks}
\begin{itemize}
    \item A state machine which given any natural language input will always return the same output where a natural language is any language which is accepted by a DFA.
    \item Any language can be used as long as every letter is defines (e.g. 0-1 (binary), a-z (english))
    \item[Note] Because DFAa are \underline{deterministic} you must define a transition state for any character in the alphabet for every state. 
    \item Sink state: where an input repeatedly returns itself into a reject state and cannot get out of it
    \item \begin{center}\includegraphics*[width=0.85\columnwidth]{DFA_phone.png}\end{center}
\end{itemize}

\section{Pigeonhole Principle}
If there exists a number $n$ of pigeonholes and $m$ pigeons is greater where $m>n$ there must be more than one pigeon sharing a hole. 

\section{Pumping Lemma for Regular Languages}

If $A$ is a regular language, then there must exist some DFA $m$ which recognizes $A$. 

Let $p$ be the number of states in $M$. 

\textit{Note} By definition the number of states must be finite. 

Consider any String $s \in A$ with length $\ge p$, there must be at least 1 state which is visited more than once when processing $s$ because of the \textit{Pigeonhole Principle}. 

Taking the path that corresponds to processing $s$ and dividing it into 3 parts, let $x$ be the portion which we cross before the first repeat, $y$ be the portion which repeats, and $z$ be the portion of the path which takes us to an accept state. 

$M$ will accept any string of the form $x{y^i}z$ for $i>0$

\subsection{Even number of 0's and 1's}
Assume machine $M$ with $p$ states recognizes the language with even numbers of 0's and 1's

Thus $s=0^P1^P$ defines the string

By the pumping lemma there must be an $x{y^i}z$ where $|y|\ge0$ and $|xy|\le p$. 

Let $i$ be arbitrarily defined as 2, $i=2$ thus $x{y^i}z=xy^2z$

We use the entire definition to recognize the portion fo 0's from the input and cannot define the portion for 1's thus contradicting the definition of a DFA. 

It is not possible to define a DFA which can recognize an even number of 0's and 1's. 

\end{document}